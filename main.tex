\documentclass{article}
\usepackage[utf8]{inputenc}
\usepackage[spanish]{babel}
\usepackage{listings}
\usepackage{graphicx}
\graphicspath{ {images/} }
\usepackage{cite}

\begin{document}

\begin{titlepage}
    \begin{center}
        \vspace*{1cm}
            
        \Huge
        \textbf{Informática II}
            
        \vspace{0.5cm}
        \LARGE
        Taller - Calistenia 
            
        \vspace{1.5cm}
            
        \textbf{Diego Fernando Urbano Palma}
            
        \vfill
            
        \vspace{0.8cm}
            
        \Large
        Despartamento de Ingeniería Electrónica y Telecomunicaciones\\
        Universidad de Antioquia\\
        Medellín\\
        Marzo de 2021
            
    \end{center}
\end{titlepage}


\section{ Agarrar la hoja de su posición inicial }\label{intro}

\section{Colocar la hoja alrededor con respecto a la posición inicial, buscando la comodidad de la persona que está realizando el ejercicio.}

\section{tomar las tarjetas que se encontraban debajo de la hoja y colocarlas encima de la hoja.}

\section{acomodar la hoja frente suyo con la finalidad de quedar cómodo.}

\section{Colocar las tarjetas una encima de la otra alineadas.}

\section{Coger las tarjetas de tal manera que el índice quede en uno de los extremos cortos de las tarjetas, el pulgar en uno de los costados largos y los otros tres dedos al otro costado respetando el paso numero 5.}

\section{ Apoyar las tarjetas sobre la hoja en el costado libre y ejercer presión sobre las tarjetas. }

\section{Soltar de manera suave la tarjeta que se encuentra contraria a la mano, sin soltar la otra tarjeta, continúa sujetando con leve presión usando el dedo índice.}

\section{Deslizar lentamente la tarjeta que tenemos sujeta, formando una V invertida con un ángulo menor o igual a 45 grados con las tarjetas.}

\section{Suelta las tarjetas después de conseguir su forma y equilibrio.}

\section{Si las tarjetas se caen repetir desde el paso 5 hasta conseguirlo.}



\end{document}
